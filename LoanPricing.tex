\documentclass{article}
\usepackage{amsmath}
\usepackage{fancyhdr}
\usepackage{amsthm}
\usepackage{amsfonts}
\usepackage{cite}
\usepackage{float}
\usepackage{scrextend}
\usepackage{hyperref}
\theoremstyle{definition}
\newtheorem{theorem}{Theorem}
\newtheorem{definition}{Definition}
\newtheorem{goal}{Goal}
\newenvironment{sketchproof}{%
  \renewcommand{\proofname}{Sketch of Proof}\proof}{\endproof}
  
\setlength{\parindent}{0pt}

\begin{document}

\section{Introduction}



\section{Mathematical preliminaries and definitions of loans}
Before diving in on pricing theory, it is useful to consider the definition and characterization of loans.  Loans involve the providing of an asset from party A (the lender) to party B (the borrower).  In return, party B pays party A over a (often predefined) set of times.  The terms of the loan typically require the sum of party B's payments to be greater than the original amount of money transferred to party B.  The excess of money that B pays A is called ``interest''.  The annualized percentage of interest relative to the original transfer is called the ``interest rate''.  

\subsection{Mathematical formulation}
Put mathematically, a loan involves an exchange of money \(P_t\) from the lender to the borrower at time \(t\).  The borrower then cumulatively pays the lender \(\int_t ^ \infty c_s ds\) for the original amount \(P_t\).  The value \(P_s\) is the par or ``book'' value of the loan.  It represents the amount of the original transfer that is owed at time \(s \in (t, T)\).
\\
\\
Typically, there is some point \(T\) which satisfies \(c_s=0\, \forall\, T<s<\infty\); that is \(T= \sup\{c_s>0 | s>=t\} \).  At this point \(T\) the value \(P_T\) is zero.  The loan is considered to be closed and neither party is beholding to the other.  
\\
\\
Another common feature of loans is that the cash flows tend to be discrete.  This enables us to use a summation instead of an integral for the cash flows.  More importantly, it allows us to model the cash flows as a series of separate (though potentially dependent) assets.  Unless specified otherwise, we will treat each cash flow independently in this book.  This enables us to define a loan as a contract in which \(P_t\) is transferred from the lender to the borrower at time \(t\) and in which the borrower promises to transfer \(c_T\) at time \(T\). For such a loan it is reasonable (though not required) to have \(c_T>P_t\).  Note that for a loan with a series of cash flows the reverse is typically true (\(c_T<P_t\)).  
\\
\\
The purpose of considering a single transfer is that it makes the exposition and mathematical derivations more clear.  It does so without losing too much generality. 

\subsubsection{A digression into yield and interest rate}

Interest rates and yields are very similar but also quite distinct.  The interest rate is the contractual rate which determines the size of the cash flows.  For example, in a simple coupon bond with \(P_t=1\) and interest rate \(5\%\) with annual cash flows for \(4\) years, the bond will be exchanged at time \(t\) for \(1\) dollar.  At years \(t+1,\,t+2, \,t+3\) the bond will pay \(5\) cents.  In the fourth year the bond will pay \(1.05\) dollars.  The interest rate stays constant over the life of this loan and is part of the original contract.  The yield is a metric used to measure the rough return of the bond.  For example, if the bond immediately after issuance started trading at \(0.98\) dollars, the yield would be higher than \(5\%\) since the coupon of \(5\) cents is being paid on a price of \(.98\) instead of on a dollar.  
\\
\\
This is even easer to see for a zero-coupon bond.  A zero coupon bond is one that pays back (without loss of generality) one dollar at time \(T\) with no intermediate payments.  An interest rate of \(5\%\) on a zero coupon bond with maturity in one year has \(P_t=e^{-.05}\approx .951\).  If this bond instantaneously started trading at \(.96\), the yield would be approximately \(4.1\%\).  
\\
\\
The difference between an initial (contractual) price or value and a value after origination will be a common theme.  
\\
\\
While interest rates and yields are convenient metrics to compare various investments and loans, we find that they confuse the narrative.  Given a value or price of a loan it is trivial to compute yields or interest rates.  This book focuses on finding the appropriate price of the loan and considers interest rates when discounting.

\subsection{Risks}
Loans introduce risk to the lender.  There are two primary risks: interest rate risk and credit risk.  Interest rate risk is the risk that alternative investments may be more attractive at some point \(\tau>t\).  However, there is often no good way to ``undo'' the loan and take advantage of alternative investments.  Credit risk is the risk that the borrower does not pay the agreed upon cash flow; that is, the borrower defaults.  
\\
\\
There are some ways to remediate or immunize these risk.  Interest rate risk can be reduced in several ways.  First, there may be conditions in the loan that allow the lender to ``call'' the loan and require immediate repayment.  These are typically rare.  Second, the loan may have a variable rate.  As market interest rates move, the payment \(c_s\) will change to accommodate.  While this helps reduce interest rate risk, it may increase credit risk since the borrower's ability to service debt may be impaired.   Third, the lender may purchase a swap to offset the interest rate risk.  Purchasing a swap brings another party to the table and may increase credit (or counterparty) risk.  
\\
\\
Credit risk can be reduced by requiring that the borrower post collateral.  In the event of default, the lender will receive the collateral.  There is still credit risk involved with securitized loans.  The collateral could depreciate to be lower than the value of the future contractual cash flows; leading to a loss.  The physical act of taking possession of the collateral may require time and resources.  
\subsubsection{Modeling Risk}
The economic definition of risk is uncertainty.  The mathematical model of uncertainty is probability.  To reflect the loan's credit risk, the cash flows from the loan can be written as 

\begin{equation} \label{riskyCashFlows} 
\int_t ^ T c_s \mathbb{I}_{\tau>s} ds+\mathbb{I}_{\tau<T} k_\tau \end{equation}
Where \(\mathbb{I}\) is the indicator function, \(\tau\) is the (random) time of default, and \(k\) is the cash flow generated from sale of collateral.  
\\
\\
In our simpler model of a single cash flow, the equation can be simplified as follows:

\begin{equation} \label{simpleRiskyCashFlows}
c_T \mathbb{I}_{\tau>T}+\mathbb{I}_{\tau<T} k_\tau
\end{equation}


\subsubsection{Risk free asset} \label{rfasset}

It is often convenient to assume the existence of a risk free asset \(M_t\).  By definition, the risk free asset returns
\[\frac{dM}{dt}=r_{f, t} M\]
Solving this ODE yields
\[M_T=M_t e^{\int_t ^T r_{f, s} ds}\]

Note that \(r_{f, t}\) is allowed to be stochastic.  Default free bonds can be considered derivative securities of this risk free asset.  The only risk in default free bonds (eg, Treasuries) is interest rate risk.  A common asset in this book will be a zero coupon bond which has a single cash flow \(c_T\).  These assets are denoted \(B(t, T)\). 

\subsection{Value through time}
\label{valOfCashFlows}
The stream of cash flows defined in \ref{riskyCashFlows} has some value at every point \(s \in (t, T)\).  This value is denoted \(V_s\).  While we have not yet developed the toolkit in this book to approach finding the ``appropriate'' value \(V_s\), intuitively it should depend, in some sense, on the cash flows defined in \ref{riskyCashFlows}.  Put mathematically, 
\[V_s=g\left(   \int_s ^ T c_u \mathbb{I}_{\tau>u} du+\mathbb{I}_{s<\tau<T} k_\tau  \right) \]

The function \(g\) may depend on other variables aside from the cashflows.  The value does not depend on cash flows that have already been paid, which is reflected in the updated lower bound of the integral.  If default has occurred before \(s\) (and collateral has been sold) \(V_s\) should be zero.
\\
\\
Note that the simpler model defined in Equation \ref{simpleRiskyCashFlows} cannot be simply added up to retrieve the actual price of a loan.  Let the following be the price or value of a single cash flow:
\[
v_j= g\left(c_{t_j} \mathbb{I}_{\tau>t_j}+\mathbb{I}_{\tau<t_j} k_\tau\right)
\]

Unless \(g\) is linear, there is no obvious method for computing \(V_s\) as a function of the \(v_j\).


\subsection{Modeling considerations: the loan market}
\label{loanMarket}
There are two markets for loans: the primary market where lenders and borrowers agree on terms for exchanging cash flows, and a secondary market where the promised cash flows can be exchanged.  For example, mortgage lenders will agree to terms with home buyers.  Mortgage lenders will then typically sell these loans to Fannie Mae.  Fannie Mae packages these loans and sells them to hedge funds, mutual funds, and other institutional investors.  
\\
\\
A secondary market does not always exist for loans which makes loan pricing both challenging and potentially rewarding. Most asset pricing literature considers that there is a market with prices for most if not all instruments.  This is the fundamental assumption for the models summarized in Section \ref{assetPricing}.  Indeed, practitioners in capital and equity markets frequently mark their models to market.  They assume a certain model for the market and then use market prices to calibrate the model.  
\\
\\
However, in general the value \(V_s\) from Section \ref{valOfCashFlows} is not a price that is viewable in the market.  This can have benefits to the lender.  If the lender believes that a borrower has better credit worthiness than the (primary or secondary) market, the lender can originate a loan at better than cost.  However, if the price is viewable in the market, the lender can make the loan ``available for sale'' (AFS).  This requires the loan to be marked to market.  Even if the lender believes the loan is higher quality, the loan will be booked as if it is lower quality with the requisite capital charge.  If the lender is right, the loan will still provide superior cash flows than expected.  This higher quality won't be reflected in the balance sheet at the time of origination. 
\\
\\
If the loan is not available for sale then the loan is booked at ``par''; that is, the value of the asset is the amount lent \(P_t\).  A simple decision for whether to originate a loan could then be 

\begin{equation}
\left\{
\begin{array}{ll}
\text{originate loan if} & V_t \geq P_t\\
\text{do not originate otherwise} 
\end{array} \right.
\end{equation}

This decision is naive.  First, a loan is required to have additional reserves for expected losses under accounting rules (SEE LATER SECTION).  Second, a loan requires capital to withstand losses that are greater than expected.  Again, this will defined more rigorously in a later section.  

\subsection{Goal of this book}

The goal of this book is introduce methods for finding the ``appropriate'' price of a loan.  Section \ref{valOfCashFlows} introduces the notion that the stream of cash flows \ref{riskyCashFlows} has some value \(V_s\) at every point in time \(s\).  The lender and the borrower agree at origination time \(t\) on the structure of the cash flows.  The implicit goal in section \ref{valOfCashFlows} is that \emph{given} a series of cash flows there is some method of computing the current value of those cash flows.  The goal for the lender at time \(t\) is to find a set of cash flows such that the value \(V_t\) is maximized.  This goal is summarized as follows:

\begin{goal}[Lender's problem] \label{lender1}
	\[\max_{c_s} V_t\]
	
\end{goal}

There are three considerations not explicitly shown in this equation which makes the goal ill-posed.  First and most importantly, the borrower is able to decline the offer.  As \(c_s\) increases, the probability of the borrower declining the offer increases.  The second consideration is that as \(c_s\) increases there is an increase in the probability that \(\tau<T\).  The borrower will have less ability to service the cash flow and is more likely to default.  The third consideration is related to both the first two considerations and to the fact that the borrower has asymmetric information about his or her ability to repay the loan: as \(c_s\) increases, any borrower that does accept the offer is more likely to be a poorer quality customer.  The fact that the borrower chose a higher \(c_s\) indicates that there is little appetite for the borrower's risk in the market.  At a very high level of \(c_s\), the borrower is likely going to take the cash and run with no intention of repayment.  This is especially the case for subprime retail portfolios.  
\\
\\
These three considerations can be captured by considering the market cash flows \(c_{m, s}\).  However, as any loan officer will say, Goal \ref{lender1} is incorrectly stated.  Rather, in a competitive (primary) market, the cash flows \(c_s\) are ``fixed'' for a given level of ``widely available'' measures of risk (say, a level of debt service coverage).  Instead of choosing the cash flow to maximize the value functions from Section \ref{valOfCashFlows}, the goal is to find areas where the lender can more accurately assess the probability that \(\tau<T\) and thus undercut the (primary) market.  
\\
\\
There are several mathematical methods for considering this phenomenon.  Perhaps the simplest is to model the ``market'' belief that the time to default random variable is \(\tau_m\) and the lender's belief is that the time to default random variable is \(\tau\).  The lender's problem then becomes the following:


\begin{goal}[Lender's problem revisited] \label{lender2}
	\[
	\left\{
	\begin{array}{ll}
	\text{originate loan if} & V_t\geq V_{t, m}\\
	\text{do not originate otherwise} 
	\end{array} \right.
	\]
	
\end{goal}

Where \(V_{t, m}\) is the market's valuation of the expected future cash flows.  Goal \ref{lender2} has the benefit of being most similar to how lenders typically run their business.  Unfortunately, goal \ref{lender2} is not academically appealing.  As seen in Section \ref{assetPricing}, models for asset pricing typically assume that the market is correct.  Hence the formulation of goal \ref{lender2} is nonsensical.  The reason that goal \ref{lender2} makes some sense is the following:


\begin{enumerate}
	\item The loan market tends to be illiquid; that is, the value \(V_t\) is not a market price but a subjective valuation of the future cash flows.
	\item Loans that are securitized and made available on the market tend to have high level risk characteristics but may have idiosyncrasies that are only visible to the lender; that is, there is asynchronous information between the lender and the market.  
	\item The market does not always trust in the competency of lenders.  This can be seen by market prices of securitizations from new lenders.  These securitizations may look the same as from established lenders, but they will typically trade at a lower value simply because there is not as much trust that the lenders have the ability to originate quality loans.  
	
\end{enumerate}

Other shortcomings with goal \ref{lender2} is that it assumes the capital structure and loan mix is identical between the market and the lender.  

\section{Introduction to Asset Pricing} \label{assetPricing}

There are three primary academic approaches to asset pricing.  The original approach is founded in utility theory.  This approach is theoretically rigorous but lacks precision.  It is best described as a phenomenological model of asset pricing but is not intended to provide a framework for precisely pricing assets.  However, the other two pricing approaches are consistent with this original approach and owe much to utility theory.  The second approach is the theory Capital Asset Pricing Model approach pioneered by Markowitz, Sharpe, and others.  Markowitz developed a theory around choosing optimal portfolios assuming either quadratic preferences or Gaussian asset returns.  Sharpe created a way to estimate what the expected return of an asset should be based off the correlation with the rest of the market.  This approach is more precise than utility theory but is prone to model error.  It is not easy to generalize the approach.  Again, the approach is primarily a good mental model for asset pricing.  The third approach is the replicating portfolio argument pioneered by Black and Scholes.  This technique is primarily applicable to derivative securities where the price can be determined by other assets.  The approach has proved very precise and rigorous.  It has also proven extensible and general.  
\\
\\
This section is devoted to more detailed descriptions of these three approaches.  The section motivates various ways of thinking about loan pricing.

\subsection{Utility theory}
Utility theory states than a given individual (homo economicus) seeks to maximize a function \(U\) that satisfies the ``usual'' conditions:
\begin{enumerate}
\item \(U'(x)>0\)
\item \(U''(x)<0\)
\end{enumerate}

In general, utility functions need not be continuous and don't require derivatives to be defined.  Indeed, all that is required for our discussion is that the utility function is concave.  However, since exact results are not required from this approach for our purposes, the useful mental picture is to assume continuity and differentiability.
\\
\\
The above conditions imply non-satiability and risk-aversion.  Risk-aversion is the more important aspect from an asset pricing perspective.  By Jenson's inequality, 
\[\mathbb{E}[\phi(X)]<=\phi(\mathbb{E}[X])\]
for concave \(\phi\).  For a risky asset \(Y\), the value placed on the asset by an individual with utility function \(U\) is \(\mathbb{E}[U(Y)]\).  This is less than the utility gained from the expected value of the asset.  The return required by the individual for the asset \(Y\) is higher than a return for a degenerate (non-risky) asset \(Z\).   
\\
\\
Applying this theory to the cash flow equation \ref{simpleRiskyCashFlows} the utility gained is

\[\mathbb{E}\left[U\left( c_T \mathbb{I}_{\tau>T}+\mathbb{I}_{\tau<T} k_\tau \right) \right]\]

If both the utility function and the distribution of \(\tau\) is known, in theory we could create the indifference curve between the utility of holding the money (or investing in a risk-free asset) and the utility of lending out the money and receiving the cash flows.  However, the utility function is only applicable to a ``representive agent'' and is very difficult to estimate in practice.  Additionally, in reality there is a large set of investments.  It is impracticable to optimize over the entire investment set.  Markowtiz solved this problem under a certain set of assumptions with modern portfolio theory.  

\subsection{Modern Portfolio Theory and CAPM}
Under the assumption that asset returns are Gaussian, Markowitz formulated an optimization problem specifying the amount to invest in every asset in the market.  The theory reduces the problem of utility optimization to one that explores the tradeoff between variance and expected returns.  For a given level of variance (risk-tolerance), the goal of the problem is to maximize the return.  The theory formalizes and makes a precise recommendation around the notion of asset diversification.  However, the theory treats expected returns and variances as fixed and known constants.  The price of the assets are not modeled; rather if every market participant followed the mean-variance recommendations there would be a market-based price for every asset based on the prevailing level of risk-tolerance.  
\\
\\
Sharpe used the theory to compute the expected return for a given stock.  To gain intuition for Sharpe's theory, consider a world in which the following is true:
\begin{enumerate}
\item There exists an infinite number of assets
\item The returns of these assets are independent (uncorrelated)
\end{enumerate}

A portfolio that contains these assets would have zero variance.  A portfolio that has no risk must return the risk free rate in equilibrium.  Using this intuition, any risky asset should return excess of the risk free rate only through its correlation with the rest of the market.  This leads to the celebrated CAPM formula 
\[\mathbb{E}[r_{i, t, T}]=r_{f, t, T}+\beta_i \left(\mathbb{E}[r_{m, t, T}]-r_{f, t, T}\right)\]
The \(\beta\) is the effect that the market has on the expected return on the asset \(i\).  A \(\beta\) of zero corresponds with an expected return of \(r_f\), or the risk free rate.
\\
\\
While loans do not have Gaussian returns, the theory still has application for loan pricing.  The primary relevance of CAPM on loan pricing is the concept of diversification and the dependence of the price of an asset on the existence of other assets.  In a large portfolio of uncorrelated loans and in an efficient market, the interest rate should only compensate for the expected loss of the loan.  
\\
\\
Let the \emph{loss distribution} of a portfolio of \(n\) loans be defined as follows:

\begin{equation}
\mathbb{P}(L<l)
\end{equation}

Where \(L=\sum_i X_i \) and \(X_i\) is the random variable describing the risky cash flows of loan \(i\).  As proven in Vasicek (1989), if \(\mathbb{E}[X_iX_j]-\mathbb{E}[X_i]\mathbb{E}[X_j]=0\,\forall\,i,\,j\) then \( \lim_{n \to \infty}\mathbb{V}[L] \to 0\).  By the linearity of expectations, the expected gain for \(L\) is simply \(\sum_i \mathbb{E}[X_i]\).  In equilibrium, this expected gain must equal the risk free rate since the portfolio has no risk.  If we further assume that the loans are homogeneous, we can also exactly find the ``correct'' price for each loan as follows: choose \(c_T\) such that

\[
\left\{c_T \,|\, \mathbb{E}\left[ c_T \mathbb{I}_{\tau>T}+\mathbb{I}_{\tau<T} k_\tau \right]=B(t, T)\right\}
\]

\subsection{Black Scholes and Risk Neutral pricing}

In 1973 Black and Scholes published their seminal paper on derivative pricing.  They showed that, under some assumptions, a certain class of derivative securities are ``redundant'' in the market; that is they can be replicated by trading in two simple assets.  This breakthrough has three primary advantages over CAPM and modern portfolio theory: it is extremely precise, does not depend on equilibrium theory (ie, it only requires the prices of a small subset of assets, and does not require estimating the expected return on an asset.  
\\
\\
Generalizations of this work show that in a complete market, the price of any derivative security can be written as an expectation.  The expectation is not under the ``real world'' probability measure but instead the ``risk-neutral'' probability measure.  
\subsubsection{Example of Risk Neutral pricing} \label{RNExample}
There have been many attempts at explaining ``risk-neutral'' pricing intuitively.  In the author's opinion, there is no simple explanation to do risk-neutral pricing justice.  In essence it is a computational trick which allows us to price using expectations.  For example, consider a simple two-state model with a stock, a risk free bond, and a call option on the stock.  The stock is currently at value \(S_0\).  The stock can be at \(S_u>S_0\) or \(S_d<S_0\) in period \(1\).  The bond is currently at value \(M\).  The value is at \(M\) in one period.  The option pays \(C_u=S_u-K\) if the stock goes up or \(C_d=0\) is the stock goes down.  
\\
\\
Consider the following strategy: construct a portfolio \(X_0\) at time \(0\) comprised of \(\Delta\) stock and \(\Gamma\) bonds.  The goal is to find \(\Delta\) and \(\Gamma\) such that at time \(1\), the final value of the portfolio is equal to the option payoff.  The problem can be solved using a simple matrix approach:

\[
\begin{bmatrix}
	M & S_u \\
	M & S_d 
\end{bmatrix}
\begin{bmatrix}
\Gamma \\
\Delta 
\end{bmatrix}
=
\begin{bmatrix}
S_u-k \\
0.0
\end{bmatrix}
\]

Using standard matrix algebra and substituting back into the formula for \(X_0\), 

\[X_0=\frac{S_0-S_d}{S_u-S_d}\left(S_u-k\right)\]

Note that \(0<\frac{S_0-S_d}{S_u-S_d}<1\) and can be interpreted as a probability.  Indeed, 

\[X_0=\tilde{p}C_u+(1-\tilde{p})C_d =\mathbb{\tilde{E}}[C]\]

Where \(\tilde{p}=\frac{S_0-S_d}{S_u-S_d}<1\).  Note that this is simply a convenient representation and has very little to do with actual probabilities of stock price movement.  The only relevance that the actual probabilities have is on the initial stock price \(S_0\).  If the actual probability of an ``up'' movement is very small then the initial stock price will likely be much closer to \(S_d\).  This will have a corresponding impact on the risk-neutral probability of an up move.  
\\
\\
By no arbitrage, the price of \(C\) at time \(0\) must be equal to \(X_0\).  Hence we have found the ``correct'' price for \(C\).  
\subsubsection{Risk-neutral pricing without a replicating portfolio}

The argument from Example \ref{RNExample} is predicated on the ability to replicate the option price with a bond and the stock.  Can a similar representation of an asset hold in the case for assets that cannot be replicated?  There seems to be some hand-wringing about this in the literature.  However, we rely on (F. Delbaen and W. Schachermayer 1998 https://people.math.ethz.ch/~delbaen/ftp/preprints/NLB.pdf) to prove the following general theorem:

\begin{theorem}[First Fundamental Theorem of Asset Pricing] \label{FFTAP}
	For a multidimensional semimartingale \(S_t\) (representing the ``market''), the following are equivalent:
	\begin{enumerate}
		\item There exists a probability measure under which \(S_t\) is a martingale
		\item The market does not admit \emph{free lunch with vanishing risk}
	\end{enumerate}
\end{theorem}

The notion of free lunch with vanishing risk essentially replaces the strong notion of no-arbitrage with the weaker notion of \(\mathcal{L}_2\) convergence.  For our purposes, the weaker notion is sufficient. 
\\
\\
Theorem \ref{FFTAP} does not provide any hints at how to price an instrument.  Indeed, if an asset cannot be replicated then there are theoretically infinite prices for the asset.  However, we follow the convention of using a measure induced by the risk neutral asset.

\subsubsection{Asset pricing without replication}

The price of an asset in this book is assumed to have the following representation:

\[\frac{X_t}{M_t}=\mathbb{\tilde{E}}\left[\frac{h(S_T)}{M_T}|\mathcal{F}_t\right]\]
Where \(S_T\) is a process that \(X_s\) depends on (eg, the stock in Example \ref{RNExample}) and \(M_t\) is the risk free asset from Section \ref{rfasset}.  Note that this representation implies that \(\frac{X_s}{M_s}\) is a martingale.  

\section{Rule for pricing a loan}

\subsection{Using the risk-neutral measure}

\subsubsection{Zero-coupon bond}
The price of a zero-coupon bond with no credit risk is the following:

\[B(t, T)=\mathbb{\tilde{E}}\left[\frac{M_t}{M_T}|\mathcal{F}_t\right]\]
\[=\mathbb{\tilde{E}}\left[e^{-\int_t ^ T r_s ds}|\mathcal{F}_t\right]\]

There are a variety of models for the instantaneous interest rate for which there are closed form solutions for \(B(t, T)\).  

\subsubsection{Basic loan with potential for default}
Consider a simple model with the risk-free asset having no return (risk free rate of zero). There is a loan that has a \(3\%\) probability of default before maturing in a year.  There is no recovery at default.  The expected value of the loan is \(0.97\).  However, the price of the loan will be less than \(0.97\) to compensate for the risk of the loan.  The risk-neutral probability of default is very simple to back out of the price: \(\hat{p}=1-S_t\) where \(\hat{p}\) is the risk-neutral probability of default and \(S_t\) is the price of the loan.  However, as mentioned in \ref{loanMarket}, there often is not a secondary market for loans.  
\\
\\
TALK ABOUT HOW EC COMPENSATES FOR RISK NEUTRAL
\subsection{Digression into optimal capital structure theory}
Firms finance their assets with debt or equity.  A common way to express this leverage (the amount of debt) is by the ratio of debt to equity (denoted henceforth as \(\alpha\)).  There has been a substantial history of literature dedicated to finding an ``optimal'' capital structure.  
\subsubsection{Modigliani and Miller}
Modigliani and Miller's (M\&M) \href{https://gvpesquisa.fgv.br/sites/gvpesquisa.fgv.br/files/arquivos/terra_-_the_cost_of_capital_corporation_finance.pdf}{paper} made the bold statement that firm value is not dependent on the leverage of an institution.  The corollary is that the expected return on equity is
\[q=\rho+(\rho-r_f)\alpha\]
Where \(q\) is the expected return on equity, \(\rho\) is the un-levered return, \(r_f\) is the usual risk free rate, and \(\alpha\) is the ratio of debt to equity.  The crux of the proof is that, under certain restrictive assumptions, investors can replicate leverage by participating in the debt markets.  These assumptions include the following:
\begin{enumerate}
	\item Zero credit risk
	\item Efficient debt markets (same access to funding for all investors)
\end{enumerate}

While the theory is an interesting academic exercise, the assumptions are extremely restrictive. However, is remains a good mental model but not for the reasons originally provided by M\&M.  As debt increases, the firm becomes more risky.  Investors require additional return on their equity to compensate for this risk.  M\&M assumed no default risk so that the risk induced by leverage is non-existent.  

\subsubsection{Extensions of M\&M}
Many of the assumptions around M\&M have been relaxed including efficient debt markets.  For example, DeMarzo's \href{https://www.gsb.stanford.edu/sites/gsb/files/publication-pdf/MM%20with%20Incomplete%20Markets%20JET%2088.pdf}{paper} shows that capital structure is irrelevant in a general equilibrium, multi-state, stochastic scenario.  However, DeMarzo still does not consider default (the model has no credit risk).

\subsubsection{Admati et al}

The theory has seen additional support recently with the \href{https://www.coll.mpg.de/pdf_dat/2010_42online.pdf}{paper} by Admati et al.  Admati et al argue that banks should be required to have substantial equity.  The costs, they argue, are small since firm value is not dependent on capital structure.  
\\
\\
The paper by Admati et al has several flaws.  First, consider a naive application of M\&M to a bank's required return on equity.  Under the assumptions of M\&M, the un-levered return of a bank is simply the risk free rate (recall there is no default in their model).  Hence the required return on equity for a bank is as follows:
\[q=r_f+(r_f-r_f)\alpha=r_f\]

In other words, the required return is the risk free rate regardless of the capital structure.  To resolve this seeming paradox, it suffices to note that under the assumptions of M\&M, banks do not exist: the debt market is efficient and has complete information.  Banks have no role as intermediaries. 
\\
\\
Another flaw in Admati et al is that they mis-represent the cause of bank defaults.  Banks do not default because of lack of equity but because of lack of liquidity.  Admati et al correctly point out that reserve requirements and capital requirements are separate considerations.  However, they spend the entire paper arguing that capital requirements are important instead of reserve requirements.  
\\
\\
The final flaw is that while they admit that debt-holders are ambivalent to the risk of firm because of government FDIC insurance, the obvious solution that is never broached is to simply remove the FDIC insurance.  This insurance simultaneously provides banks with cheap funding and little incentive to hold capital.  It also acts as a barrier to entry for non-banks.  Removing it would force banks to become less levered.  It would force banks to hold more reserves.  
\\
\\
One final remark about Admati et al.  They make the following statement regarding M\&M versus other models:
\\
\\
\begin{addmargin}[2em]{2em}
The assumptions underlying the Modigliani-Miller analysis are in fact the very same assumptions underlying the quantitative models that  banks  use  to  manage  their risks, in particular, the risks in their trading books. Anyone who questions the empirical validity and relevance of an analysis that is based on these assumptions is implicitly questioning the reliability of these quantitative models and their adequacy for the uses to which they are put – including that of determining required capital under the model-based approach for market risk.
\\
\\
\end{addmargin}

This statement is a gross mis-characterization of market models.  First, market models used by banks use the much more precise and rigorous foundation of risk-neutral pricing rather than the more descriptive equilibrium models of M\&M.  Secondly, market models are marked-to-market: these models are not necessarily intended to be accurate in and of themselves, but instead transform market signals into more stable metrics (for example, implied volatility).  Third, certain models are more robust to their assumptions than others.  Even if M\&M shared similar assumptions to market models, M\&M is clearly less robust to these assumptions than market models.  In a rather ironic twist, the acceptance of market models by the market speaks to the utility of these models over models like M\&M.  

\subsubsection{DeAngelo and Stultz}

On the other side of the debate is DeAngelo and Stultz's \href{https://fic.wharton.upenn.edu/wp-content/uploads/2016/11/13-20.pdf}{paper} on optimal bank leverage.  Their paper discusses various relaxations of the M\&M assumptions.  First, they consider relaxations of M\&M such that banks can exist while still retaining the M\&M result.  Second, they consider liquidity generation to be a primary banking function.  The first point is relevant to our discussion on loan pricing.  The second is relevant for finding socially optimal capital structures.  However, it less relevant for pricing individual loans.


\subsubsection{Capital structure as option pricing}

While the debates around M\&M are fun to read and participate in, the results are not precise or general enough for our purposes.  We need something which incorporates the probability of default; both in the riskiness of the assets of a lender and of the lender itself.  

Merton wrote a \href{http://www.people.hbs.edu/rmerton/Pricing%20of%20corporate%20debt.pdf}{paper} showing that equity is similar to a long call option an a firm's assets while debt is similar to a short put option on a firm's assets.  


\subsection{Putting it all together}


There are now four formulas or values.  These are

\begin{enumerate}
	\item \(V_t\): the modeled (subjective) value of the loan to the lender
	\item \(V_{t, min}\): the modeled (subjective) minimum value of the loan to the lender
	\item \(V_{t, m}\): the (directly unobservable) value of the loan in the primary market
	\item \(S_t\): the (potentially unobservable) price of the loan in the secondary market
	
\end{enumerate}

The lending decision should be:


\begin{goal}[Lender's final problem] \label{lender3}
	\[
	\left\{
	\begin{array}{ll}
	\text{originate loan if} & V_t \geq V_{t, min}\\
	\text{do not originate otherwise} 
	\end{array} \right.
	\]
	
\end{goal}

The primary and secondary market values do not directly impact the origination decision.  However, these values should be tracked over a broad set of originations.  For example, consider the following scenarios:

\begin{enumerate}
	\item ``Good'' scenario:  \(V_t<S_t\) or \(V_t<V_{t, m}\).  If the lender's subjective valuation is consistently lower than the market's, the lender's capital is either better suited (or more diversified) for the loans that the lender originates or the lender's models are better (or worse!) than the market's.  The performance of the model can be tracked temperately.  If the model performance is adequate, then consistently undercutting the  market is a good thing.  
	\item ``Bad'' scenario: \(V_{t, min}>S_t\) or \(V_{t, min}>V_{t, m}\).  The lender is inappropriately deploying capital or has a significant model problem.  The lender may need to consider other lending products.  
\end{enumerate}

These values can be tracked in a number of ways.  The primary market valuation for a loan can be assessed through an ongoing study of percentage of applications are funded and through random price testing.  The secondary market price can only be tracked for those loans that are publicly tradeable.  Secondary market loans may have less information about the borrower than is known by the lender which may influence the price.  
\\
\\
TODO! Proof that goal \ref{lender3} will achieve the correct price in an efficient market.  


\end{document}